\chapter{Cases}
\label{chapter: cases}

\section{How to run cases}
Only two files are needed to set up and run a case with the SCM. The first is a configuration namelist file found in \execout{gmtb-scm/scm/etc/case\_config}. Each case configuration file contains two fortran namelists, one called \execout{case\_config} that contains parameters for the SCM infrastructure and one called \execout{physics\_config} that contains parameters for the physics suite(s). The second necessary file is a netCDF file containing data to initialize the column state and time-dependent data to force the column state. The two files are described below.

\subsection{Case configuration namelist parameters}
\label{subsection: case config}
The \execout{case\_config} namelist expects the following parameters:
\begin{itemize}
\item \execout{model\_name}
	\begin{itemize}
	\item This controls which vertical coordinates to use. Valid values are \exec{'FV3'} or \exec{`GFS'}. Here, \exec{`GFS'} refers to vertical coordinates used in the GSM.
	\end{itemize}
\item \execout{n\_columns}
	\begin{itemize}
	\item The code can be used to run a single column or multiple \emph{independent} columns using the same or different physics suites. Specify an integer, \exec{n}. NOTE: As of this release, only \execout{n\_columns} $= 1$ is supported.
	\end{itemize}
\item \execout{case\_name}
	\begin{itemize}
	\item Identifier for which dataset (initialization and forcing) to load. This string must correspond to a dataset included in the directory \execout{gmtb-scm/scm/data/processed\_case\_input/} (without the file extension).
	\end{itemize}
\item \execout{dt}
	\begin{itemize}
	\item Time step in seconds (floating point)
	\end{itemize}
\item \execout{time\_scheme}
	\begin{itemize}
	\item Specify 1 for the forward-Euler time-stepping scheme or 2 for the filtered leapfrog scheme.
	\end{itemize}
\item \execout{runtime}
	\begin{itemize}
	\item Specify the model runtime in seconds (integer). This should correspond with the forcing dataset used. If a runtime is specified that is longer than the supplied forcing, the forcing is held constant at the last specified values.
	\end{itemize}
\item \execout{output\_frequency}
	\begin{itemize}
	\item Specify the frequency of the model output in seconds (floating point). Currently, no averaging of the output fields is done if \execout{output\_frequency} $\neq$ \execout{dt}; only instantaneous output at the supplied frequency is implemented.
	\end{itemize}
\item \execout{n\_levels}
	\begin{itemize}
	\item Specify the integer number of vertical levels. If \execout{model\_name}=\exec{`GFS'}, only values of 28, 42, 60, 64, 91 are supported.
	\end{itemize}
\item \execout{output\_dir}
	\begin{itemize}
	\item A string representing the path (relative to the build directory) to which output should be written.
	\end{itemize}
\item \execout{output\_file}
	\begin{itemize}
	\item A string representing the name of the netCDF output file to be written (no \exec{.nc} extension expected).
	\end{itemize}
\item \execout{case\_data\_dir}
	\begin{itemize}
	\item A string representing the path (relative to the build directory) where case initialization and forcing data files can be found.
	\end{itemize}
\item \execout{vert\_coord\_data\_dir}
	\begin{itemize}
	\item A string representing the path (relative to the build directory) where vertical coordinate data files can be found (for \execout{model\_name}=\exec{`GFS'} only).
	\end{itemize}
\item \execout{thermo\_forcing\_type}
	\begin{itemize}
	\item An integer representing how forcing for temperature and moisture state variables is applied (1 $=$ total advective tendencies, 2 $=$ horizontal advective tendencies with prescribed vertical motion, 3 $=$ relaxation to observed profiles with vertical motion prescribed)
	\end{itemize}
\item \execout{mom\_forcing\_type}
	\begin{itemize}
	\item An integer representing how forcing for horizontal momentum state variables is applied (1 $=$ total advective tendencies; not implemented yet, 2 $=$ horizontal advective tendencies with prescribed vertical motion, 3 $=$ relaxation to observed profiles with vertical motion prescribed)
	\end{itemize}
\item \execout{relax\_time}
	\begin{itemize}
	\item A floating point number representing the timescale in seconds for the relaxation forcing (only used if \execout{thermo\_forcing\_type} $=$ \exec{3} or \execout{mom\_forcing\_type} $=$ \exec{3})
	\end{itemize}
\item \execout{sfc\_flux\_spec}
	\begin{itemize}
	\item A boolean set to \exec{.true.} if surface flux are specified from the forcing data (there is no need to have surface schemes in a suite definition file if so)
	\end{itemize}
\item \execout{sfc\_type}
	\begin{itemize}
	\item An integer representing the character of the surface (0 $=$ sea surface, 1 $=$ land surface, 2 $=$ sea-ice surface)
	\end{itemize}
\item \execout{reference\_profile\_choice}
	\begin{itemize}
	\item An integer representing the choice of reference profile to use above the supplied initialization and forcing data (1 $=$ ``McClatchey'' profile, 2 $=$ mid-latitude summer standard atmosphere)
	\end{itemize}
\item \execout{year}
	\begin{itemize}
	\item An integer representing the year of the initialization time
	\end{itemize}
\item \execout{month}
	\begin{itemize}
	\item An integer representing the month of the initialization time
	\end{itemize}
\item \execout{day}
	\begin{itemize}
	\item An integer representing the day of the initialization time
	\end{itemize}
\item \execout{hour}
	\begin{itemize}
	\item An integer representing the hour of the initialization time
	\end{itemize}
\end{itemize}

\subsection{Physics configuration namelist parameters}
\label{subsection: physics config}

The \execout{physics\_config} namelist expects the following parameters:

\begin{itemize}
\item \execout{physics\_suite}
	\begin{itemize}
	\item A string list representing the names of the physics suite for each column (must correspond a suite definition file name; if using multiple columns, you may specify an equal number of physics suites)
	\end{itemize}
\item \execout{physics\_suite\_dir}
	\begin{itemize}
	\item A string representing the path (relative to the build directory) where suite definition files can be found.
	\end{itemize}
\item \execout{physics\_nml}
	\begin{itemize}
	\item A string list representing the paths (relative to the build directory) to the physics namelist files.
	\end{itemize}
\item \execout{column\_area}
	\begin{itemize}
	\item A list of floating point values representing the characteristic horizontal domain area of each atmospheric column in square meters (this could be analogous to a 3D model's horizontal grid size or the characteristic horizontal scale of an observation array; these values are used in scale-aware schemes; if using multiple columns, you may specify an equal number of column areas)
	\end{itemize}
\end{itemize}

\subsection{Case input data file}
\label{subsection: case input}

The initialization and forcing data for each case is stored in a netCDF (version 4) file within the \execout{gmtb-scm/scm/data/processed\_case\_input} directory. Each file has two dimensions (\execout{time} and \execout{levels}) and is organized into 3 groups: scalars, initial, and forcing. Not all fields are required for all cases. For example the fields \execout{sh\_flux\_sfc} and \execout{lh\_flux\_sfc} are only needed if the variable \execout{sfc\_flx\_spec} $=$ \exec{.true.} in the case configuration file and state nudging variables are only required if \execout{thermo\_forcing\_type} $=$ \exec{3} or \execout{mom\_forcing\_type} $=$ \exec{3}.

\lstinputlisting[
                 basicstyle=\scriptsize\ttfamily,
                 label=lst_case_input_netcdf_header,
                 caption=example netCDF file header for case initialization and forcing data
                 ]{./arm_case_header.txt}

\section{How to set up new cases}

Setting up a new case involves preparing the two types of files listed above. For the case initialization and forcing data file, this typically involves writing a custom script or program to parse the data from its original format to the format that the SCM expects, listed above. An example of this type of script written in python is included in \execout{/gmtb-scm/scm/etc/scripts/twpice\_forcing\_file\_generator.py}. The script reads in the data as supplied from its source, converts any necessary variables, and writes a netCDF (version 4) file in the format described in subsection \ref{subsection: case input}. For reference, the following formulas are used:
\begin{equation}
\theta_{il} = \theta - \frac{\theta}{T}\left(\frac{L_v}{c_p}q_l + \frac{L_s}{c_p}q_i\right)
\end{equation}
\begin{equation}
q_t = q_v + q_l + q_i
\end{equation}
where $\theta_{il}$ is the ice-liquid water potential temperature, $\theta$ is the potential temperature, $L_v$ is the latent heat of vaporization, $L_s$ is the latent heat of sublimation $c_p$ is the specific heat capacity of air at constant pressure, $T$ is absolute temperature, $q_t$ is the total water specific humidity, $q_v$ is the water vapor specific humidity, $q_l$ is the suspended liquid water specific humidity, and $q_i$ is the suspended ice water specific humidity.

As shown in the example netCDF header, the SCM expects that the vertical dimension is pressure levels (index 1 is the surface) and the time dimension is in seconds. The initial conditions expected are the height of the pressure levels in meters, and arrays representing vertical columns of $\theta_{il}$ in K, $q_t$, $q_l$, and $q_i$ in kg kg$^{-1}$, $u$ and $v$ in m s$^{-1}$, turbulence kinetic energy in m$^2$ s$^{-2}$ and ozone mass mixing ratio in kg kg$^{-1}$.

For forcing data, the SCM expects a time series of the following variables: latitude and longitude in decimal degrees [in case the column(s) is moving in time (e.g., Lagrangian column)], the surface pressure (Pa) and surface temperature (K). If surface fluxes are specified for the new case, one must also include a time series of the kinematic surface sensible heat flux (K m s$^{-1}$) and kinematic surface latent heat flux (kg kg$^{-1}$ m s$^{-1}$). The following variables are expected as 2-dimensional arrays (vertical levels first, time second):  the geostrophic u (E-W) and v (N-S) winds (m s$^{-1}$), and the horizontal and vertical advective tendencies of $\theta_{il}$ (K s$^{-1}$) and $q_t$ (kg kg$^{-1}$ s$^{-1}$), the large scale vertical velocity (m s$^{-1}$), large scale pressure vertical velocity (Pa s$^{-1}$), the prescribed radiative heating rate (K s$^{-1}$), and profiles of u, v, T, $\theta_{il}$ and $q_t$ to use for nudging.

Although it is expected that all variables are in the netCDF file, only those that are used with the chosen forcing method are required to be nonzero. For example, the following variables are required depending on the values of \execout{mom\_forcing\_type} and \execout{thermo\_forcing\_type} specified in the case configuration file:

\begin{itemize}
\item \execout{mom\_forcing\_type} $=$ \exec{1}
	\begin{itemize}
		\item Not implemented yet
	\end{itemize}
\item \execout{mom\_forcing\_type} $=$ \exec{2}
	\begin{itemize}
		\item geostrophic winds and large scale vertical velocity
	\end{itemize}
\item \execout{mom\_forcing\_type} $=$ \exec{2}
	\begin{itemize}
		\item u and v nudging profiles
	\end{itemize}
\item \execout{thermo\_forcing\_type} $=$ \exec{1}
	\begin{itemize}
		\item horizontal and vertical advective tendencies of $\theta_{il}$ and $q_t$ and prescribed radiative heating (can be zero if radiation scheme is active)
	\end{itemize}
\item \execout{thermo\_forcing\_type} $=$ \exec{2}
	\begin{itemize}
		\item horizontal advective tendencies of $\theta_{il}$ and $q_t$, prescribed radiative heating (can be zero if radiation scheme is active), and the large scale vertical pressure velocity
	\end{itemize}
\item \execout{thermo\_forcing\_type} $=$ \exec{2}
	\begin{itemize}
		\item $\theta_{il}$ and $q_t$ nudging profiles and the large scale vertical pressure velocity
	\end{itemize}
\end{itemize}

For the case configuration file, it is most efficient to copy an existing file in \execout{gmtb-scm/scm/etc/case\_config} and edit it to suit one's case. Recall from subsections \ref{subsection: case config} and \ref{subsection: physics config} that this file is used to configure both the SCM framework parameters (including how forcing is applied) and some physics suite parameters. Be sure to check that model timing parameters such as the time step and output frequency are appropriate for the physics suite being used. There is likely some stability criterion that governs the maximum time step based on the chosen parameterizations and number of vertical levels (grid spacing). The \execout{case\_name} parameter should match the name of the case input data file that was configured for the case (without the file extension). The \execout{runtime} parameter should be less than or equal to the length of the forcing data unless the desired behavior of the simulation is to proceed with the last specified forcing values after the length of the forcing data has been surpassed. The initial date and time should fall within the forcing period specified in the case input data file. If the case input data is specified to a lower altitude than the vertical domain, the remainder of the column will be filled in with values from a reference profile. There is a tropical profile and mid-latitude summer profile provided, although one may add more choices by adding a data file to \execout{gmtb-scm/scm/data/processed\_case\_input} and adding a parser section to the subroutine \execout{get\_reference\_profile} in \execout{gmtb-scm/scm/src/gmtb\_scm\_input.f90}. Surface fluxes can either be specified in the case input data file or calculated using a surface scheme using surface properties. If surface fluxes are specified from data, set \execout{sfc\_flux\_spec} to \exec{.true.} and specify \execout{sfc\_roughness\_length\_cm} for the surface over which the column resides. Otherwise, specify a \execout{sfc\_type}.

To control the forcing method, one must choose how the momentum and scalar variable forcing are applied. The three methods of Randall and Cripe (1999, JGR) have been implemented: ``revealed forcing'' where total (horizontal $+$ vertical) advective tendencies are applied (type 1), ``horizontal advective forcing'' where horizontal advective tendencies are applied and vertical advective tendencies are calculated from a prescribed vertical velocity and the calculated (modeled) profiles (type 2), and ``relaxation forcing'' where nudging to observed profiles replaces horizontal advective forcing combined with vertical advective forcing from prescribed vertical velocity (type 3). If relaxation forcing is chosen, a \execout{relaxation\_time} that represents the timescale over which the profile would return to the nudging profiles must be specified.

The \execout{physics\_config} namelist portion of the case configuration file provides a place to specify which physics are called. One needs to specify the path where the CCPP suite definition file is located in the \execout{physics\_suite\_dir} parameter. If one has specified that more than one (independent) columns be simulated, one also need to specify the name of the physics suite to be used for each column (corresponding to a suite definition file without the file extension) and a namelist that contains physics suite parameters for each column. The physics suite parameter namelist files are read as part of the suite initialization subroutine specified in the suite definition file. In addition, one must specify a \execout{column\_area} for each column. As of this release, multiple column functionality is limited to changes in the physics -- one can run multiple different physics suites using the same initial conditions and forcing, perform sensitivity test through multiple physics suite parameter namelists, and/or sensitivity tests with different representative column areas.

\section{Using other LASSO cases}
\label{sec:lasso}

In order to use other LASSO cases than the one provided, perform the following steps:
\begin{enumerate}
\item Access \url{http://archive.arm.gov/lassobrowser} and use the navigation on the left to choose the dates for which you would like to run a SCM simulation. Pay attention to the ``Large Scale Forcing'' tab where you can choose how the large scale forcing was generated, with options for ECMWF, MSDA, and VARANAL. All are potentially valid, and it is likely worth exploring the differences among forcing methods. Click on Submit to view a list of simulations for the selected criteria. Choose from the simulations (higher skill scores are preferred) and check the ``Config Obs Model Tar'' box to download the data. Once the desired simulations have been checked, order the data (you may need to create an ARM account to do so).
\item Once the data is downloaded, decompress it. From the \execout{config} directory, copy the files \execout{input\_ls\_forcing.nc}, \execout{input\_sfc\_forcing.nc}, and \execout{wrfinput\_d01.nc} into their own directory under \execout{gmtb-scm/scm/data/raw\_case\_input/}.
\item Modify \execout{gmtb-scm/scm/etc/scripts/lasso1\_forcing\_file\_generator\_gjf.py} to point to the input files listed above. Execute the script in order to generate a case input file for the SCM (to be put in \execout{gmtb-scm/scm/data/processed\_case\_input/}):
\begin{lstlisting}[language=bash]
./lasso1_forcing_file_generator_gjf.py
\end{lstlisting}
\item Create a new case configuration file (or copy and modify an existing one) in \execout{gmtb-scm/scm/etc/case\_config}. Be sure that the \execout{case\_name} variable points to the newly created/processed case input file from above.
\end{enumerate}
