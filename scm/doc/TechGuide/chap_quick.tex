\chapter{Quick Start Guide}
\label{chapter: quick}

This chapter provides instructions for obtaining and compiling  the GMTB SCM. The SCM code calls CCPP-compliant physics schemes through the CCPP framework code. As such, it requires the CCPP framework code and physics code, both of which are included as subdirectories within the SCM code. This package can be considered a simple example for an atmospheric model to interact with physics through the CCPP.

\section{Obtaining Code}

The source code bundle for the CCPP and SCM is provided through GitHub.com.  This release repository contains the tested and supported version for general use.

\begin{enumerate}
    \item Download a compressed file or clone the source using
\begin{lstlisting}[language=bash]
git clone https://[username]@github.com/NCAR/gmtb-scm-release gmtb-scm
\end{lstlisting}
    and enter your github password when prompted.
    \item Check out the desired \hl{release number} using
\begin{lstlisting}[language=bash]
git checkout [v2.0]
\end{lstlisting}
    \item Change directory into the project.
\begin{lstlisting}[language=bash]
cd gmtb-scm
\end{lstlisting}
\end{enumerate}

The CCPP framework can be found in the ccpp/framework subdirectory at this level.  The CCPP physics parameterizations can be found in the ccpp/physics subdirectory.

If you would like to contribute as a developer to this project, please see CCPP Developers Corner of the project website:

\url{https://dtcenter.org/gmtb/users/ccpp/developers/index.php}

There you will find links to all of the documentation pertinent to developers.

One can also clone the development repository, although its stability is not guaranteed. To do so,
\begin{enumerate}
    \item Clone the source using
\begin{lstlisting}[language=bash]
git clone --recursive https://[username]@github.com/NCAR/gmtb-scm gmtb-scm
\end{lstlisting}
    and enter your github password when prompted.
    \item Check out the desired branch (master) using
\begin{lstlisting}[language=bash]
git checkout master
\end{lstlisting}
    \item Change directory into the project.
\begin{lstlisting}[language=bash]
cd gmtb-scm
\end{lstlisting}
\end{enumerate}


\section{System Requirements, Libraries, and Tools}
\label{section: systemrequirements}

The source code for the SCM and CCPP component is in the form of programs written in FORTRAN, FORTRAN 90, and C. In addition, the I/O relies on the netCDF libraries. Beyond the standard scripts, the build system relies on use of the Python scripting language, along with cmake, GNU make and date.

The basic requirements for building and running the CCPP and SCM bundle are \hl{listed below}:
\begin{itemize}
    \item FORTRAN 90+ compiler (ifort v17+, gfortran v5.4+, pgf90 v17.9+)
    \item C compiler (icc v17+, gcc v5.4+, pgcc v17.9+)
    \item cmake v2.8.11+
    \item netCDF v4.x (not v3.x) with HDF5, ZLIB and SZIP
    \item Python v2.x (not v3.x)
    \item Libxml2 (tested with 2.2 and 2.9.1)
\end{itemize}

Because these tools and libraries are typically the purview of system administrators to install and maintain, they are considered  part of the basic system requirements.

There are several utility libraries provided in the SCM bundle, as external packages. These are built during the compilation phase, and include
\begin{itemize}
    \item bacio - Binary I/O Library
    \item sp - Spectral Transformation Library
    \item w3nco - GRIB decoder and encoder library
\end{itemize}

\subsection{Compilers}
The CCPP and SCM have been tested on a variety of
computing platforms. Currently the CCPP system is actively supported
on Linux and MacOS computing platforms using the Intel, PGI or GNU Fortran
compilers. Please use versions listed in the previous section as unforeseen
build issues may occur when using older compiler versions. Typically the best results come from using the
most recent version of a compiler. If you have problems with compilers, please check the ``Known Issues'' section of the
community website (\url{https://dtcenter.org/gmtb/users/ccpp/support/CCPP_KnownIssues.php}).

\section{Compiling SCM with CCPP}
\label{section: compiling}
The first step in compiling the CCPP and SCM is to match the physics variables (between what the host model -- SCM -- can provide and what is needed by physics schemes in the CCPP), and build the physics caps needed to use them.  Following this step, the top level build system will use \execout{cmake} to query system parameters and \execout{make} to compile the components.  A platform-specific script is provided to load modules and set the user environment for common platforms.  If you are not using one of these platforms, you will need to set up the same environment on your platform.
\begin{enumerate}
    \item Run the CCPP prebuild script to match required physics variables with those available from the dycore (SCM) and to generate physics caps and makefile segments.
\begin{lstlisting}[language=bash]
./ccpp/framework/scripts/ccpp_prebuild.py --config=./ccpp/config/ccpp_prebuild_config.py [--debug]
\end{lstlisting}
    \item Change directory to the top-level SCM directory.
\begin{lstlisting}[language=bash]
cd scm
\end{lstlisting}
    \item (Optional) Run the machine setup script if necessary. This script loads compiler modules (Fortran 2003-compliant), netCDF module, etc. and sets compiler environment variables. For \textit{t/csh} shells,
\begin{lstlisting}[language=csh]
source etc/Theia_setup_gnu.csh
source etc/Theia_setup_intel.csh
source etc/Theia_setup_pgi.csh
source etc/Cheyenne_setup_gnu.csh
source etc/Cheyenne_setup_intel.csh
source etc/Cheyenne_setup_pgi.csh
source etc/UBUNTU_setup.csh
source etc/CENTOS_setup.csh
source etc/MACOSX_setup.csh
\end{lstlisting}
For bourne/bash shells,
\begin{lstlisting}[language=bash]
. etc/Theia_setup_gnu.sh
. etc/Theia_setup_intel.sh
. etc/Theia_setup_pgi.sh
. etc/Cheyenne_setup_gnu.sh
. etc/Cheyenne_setup_intel.sh
. etc/Cheyenne_setup_pgi.sh
. etc/UBUNTU_setup.sh
. etc/CENTOS_setup.sh
. etc/MACOSX_setup.sh
\end{lstlisting}
\emph{Note:} If using a local Linux or Mac system, we provide instructions for how to set up your development system (compilers and libraries) in \execout{doc/README\_\{MACOSX,UBUNTU,CENTOS\}.txt}. If following these, you will need to run the respective setup script listed above. If your computing environment was previously set up to use modern compilers with an associated netCDF installation, it may not be necessary, although we recommend setting environment variables such as \execout{CC} and \execout{FC}. \textbf{For version 3.0 and above, it is required to have the \execout{NETCDF} environment variable set to the path of the netCDF installation that was compiled with the same compiler used in the following steps}. Otherwise, the \execout{cmake} step will not complete successfully.

    \item Make a build directory and change into it.
\begin{lstlisting}[language=bash]
mkdir bin && cd bin
\end{lstlisting}
    \item Invoke cmake on the source code to build.
\begin{lstlisting}[language=bash]
cmake ../src                          # without threading/OpenMP
cmake -DOPENMP=ON ../src               # with threading/OpenMP
cmake -DCMAKE_BUILD_TYPE=Debug ../src # debug mode
\end{lstlisting}
    \item If cmake cannot find \execout{libxml2} because it is installed in a non-standard location, add
\begin{lstlisting}[language=bash]
-DPC_LIBXML_INCLUDEDIR=... -DPC_LIBXML_LIBDIR=...
\end{lstlisting}
    to the cmake command.
    \item Compile. Add \execout{VERBOSE=1} to obtain more information on the build process.
\begin{lstlisting}[language=bash]
make
\end{lstlisting}
\end{enumerate}

The resulting executable may be found at \execout{./gmtb-scm} (Full path of \execout{gmtb-scm/scm/bin/gmtb-scm}). Depending on the system, it may be necessary to add the location of the CCPP framework and physics libraries to \execout{LD\_LIBRARY\_PATH} to run \execout{./gmtb-scm} (see next section).

If you encounter errors, please capture a log file from all of the steps, and contact the helpdesk at: \url{gmtb-help@ucar.edu}

\section{Run the SCM with a supplied case}
There are several test cases provided with this version of the SCM. For all cases, the SCM will go through the time steps, applying forcing and calling the physics defined in the chosen suite definition file using physics configuration options from an associated namelist. The model is executed through one of two python run scripts that are pre-staged into the \execout{bin} directory: \execout{run\_gmtb\_scm.py} or \execout{multi\_run\_gmtb\_scm.py}. The first sets up and runs one integration while the latter will set up and run several integrations serially. 

\subsection{Single Run Script Usage}
Running a case requires three pieces of information: the case to run (consisting of initial conditions, geolocation, forcing data, etc.), the physics suite to use (through a CCPP suite definition file), and a physics namelist (that specifies configurable physics options to use). As discussed in chapter \ref{chapter: cases}, cases are set up via their own namelists in \execout{../etc/case\_config}. A default physics suite is provided as a user-editable variable in the script and default namelists are associated with each physics suite (through \execout{../src/default\_namelists.py}), so, technically, one must only specify a case to run with the SCM. The single run script's interface is described below.

\begin{lstlisting}[language=bash]
./run_gmtb_scm.py -c CASE_NAME [-s SUITE_NAME] [-n PHYSICS_NAMELIST_PATH] [-g]
\end{lstlisting}

When invoking the run script, the only required argument is the name of the case to run. The case name used must match one of the case configuration files located in \execout{../etc/case\_config} (\emph{without the .nml extension!}). If specifying a suite other than the default, the suite name used must match the value of the suite name in one of the suite definition files located in \execout{../../ccpp/suites} (Note: not the filename of the suite definition file). As part of the third CCPP release, the following suite names are valid:
\begin{enumerate}
\item SCM\_GFS\_v15
\item SCM\_GFS\_v15plus
\item SCM\_csawmg
\item SCM\_GSD\_v0
\end{enumerate}

Note that using the Thompson microphysics scheme (as in SCM\_GSD\_v0) requires the computation of look-up tables during its initialization phase. As of the release, this process has been prohibitively slow with this model, so it is HIGHLY suggested that these look-up tables are downloaded and staged to use this scheme (and the SCM\_GSD\_v0 suite). Pre-computed tables have been created and are available for download at the following URLs:
\begin{itemize}
\item \url{https://dtcenter.org/GMTB/freezeH2O.dat} (243 M)
\item \url{https://dtcenter.org/GMTB/qr_acr_qg.dat} (49 M)
\item \url{https://dtcenter.org/GMTB/qr_acr_qs.dat} (32 M)
\end{itemize}
These files should be staged in \execout{gmtb-scm/scm/data/GFS\_physics\_data} prior to running \execout{cmake}. Since binary files can be system-dependent (due to endianness), it is possible that these files will not be read correctly on your system. For reference, the linked files were generated on Theia using the Intel v18 compiler.

Also note that some cases require specified surface fluxes. Special suite definition files that correspond to the suites listed above have been created and use the \execout{*\_prescribed\_surface} decoration. It is not necessary to specify this filename decoration when specifying the suite name. If the \execout{spec\_sfc\_flux} variable in the configuration file of the case being run is set to \execout{.true.}, the run script will automatically use the special suite definition file that corresponds to the chosen suite from the list above.

If specifying a namelist other than the default, the value must be an entire filename that exists in \execout{../../ccpp/physics\_namelists}. Caution should be exercised when modifying physics namelists since some redundancy between flags to control some physics parameterizations and scheme entries in the CCPP suite definition files currently exists. Values of numerical parameters are typically OK to change without fear of inconsistencies. Lastly, the \execout{-g} flag can be used to run the executable through the \exec{gdb} debugger (assuming it is installed on the system).

If the run aborts with the error message
\begin{lstlisting}[language=bash]
gmtb_scm: libccppphys.so.X.X.X: cannot open shared object file: No such file or directory
\end{lstlisting}
the environment variable \execout{LD\_LIBRARY\_PATH} must be set to
\begin{lstlisting}[language=bash]
export LD_LIBRARY_PATH=$PWD/ccpp/physics:$LD_LIBRARY_PATH
\end{lstlisting}
before running the model.

A netCDF output file is generated in the location specified in the case
configuration file, if the \execout{output\_dir} variable exists in that file. Otherwise an output directory is constructed from the case, suite, and namelist used (if different from the default). All output directories are placed in the \execout{bin} directory. Any standard netCDF file viewing or analysis tools may be used to
examine the output file (ncdump, ncview, NCL, etc).

\subsection{Multiple Run Script Usage}

A second python script is provided for automating the execution of multiple integrations through repeated calling of the single run script. From the run directory, one may use this script through the following interface.

\begin{lstlisting}[language=bash]
./multi_run_gmtb_scm.py {[-c CASE_NAME] [-s SUITE_NAME] [-f PATH_TO_FILE]} [-v{v}] [-t]
\end{lstlisting}

No arguments are required for this script. The \execout{-c or --case}, \execout{-s or --suite}, or \execout{-f or --file} options form a mutually-exclusive group, so exactly one of these is allowed at one time. If \execout{--c} is specified with a case name, the script will run a set of integrations for all supported suites (defined in \execout{../src/supported\_suites.py}) for that case. If \execout{-s} is specified with a suite name, the script will run a set of integrations for all supported cases (defined in \execout{../src/supported\_cases.py}) for that that suite. If \execout{-f} is specified with the path to a filename, it will read in lists of cases, suites, and namelists to use from that file. If multiple namelists are specified in the file, there either must be one suite specified \emph{or} the number of suites must match the number of namelists. If none of the \execout{-c or --case}, \execout{-s or --suite}, or \execout{-f or --file} options group is specified, the script will run through all permutations of supported cases and suites (as defined in the files previously mentioned).

In addition to the main options, some helper options can also be used with any of those above. The \execout{-v{v} or --verbose} option can be used to output more information from the script to the console and to a log file. If this option is not used, only completion progress messages are written out. If one \execout{-v} is used, the script will write out completion progress messages and all messages and output from the single run script. If two \execout{-vv} are used, the script will also write out all messages and single run script output to a log file (\execout{multi\_run\_gmtb\_scm.log}) in the \execout{bin} directory. The final option, \execout{-t or --timer}, can be used to output the elapsed time for each integration executed by the script. Note that the execution time includes file operations performed by the single run script in addition to the execution of the underlying (Fortran) SCM executable. By default, this option will execute one integration of each subprocess. Since some variability is expected for each model run, if greater precision is required, the number of integrations for timing averaging can be set through an internal script variable. This option can be useful, for example, for getting a rough idea of relative computational expense of different physics suites.

\subsection{Batch Run Script}

If using the model on HPC resources and significant amounts of processor time is anticipated for the experiments, it will likely be necessary to submit a job through the HPC's batch system. An example script has been included in the repository for running the model on Theia's batch system (SLURM). It is located in \execout{gmtb-scm/scm/etc/gmtb\_scm\_slurm\_example.py}. Edit the \execout{job\_name}, \execout{account}, etc. to suit your needs and copy to the \execout{bin} directory. The case name to be run is included in the \execout{command} variable. To use, invoke
\begin{lstlisting}[language=bash]
./gmtb_scm_slurm_example.py
\end{lstlisting}
from the \execout{bin} directory.

Additional details regarding the SCM may be found in the remainder of this guide. More information on the CCPP can be found in the CCPP Developers' Corner available at \url{https://dtcenter.org/gmtb/users/ccpp/developers} and in the CCPP Developers' Guide at \url{https://dtcenter.org/gmtb/users/ccpp/docs}.
